
Wie Ihr dem folgenden Stundenplan entnehmen könnt, enthält das Medieninformatik-Studium im ersten
Semester neben den Informatikvorlesungen auch eine gute Portion Mathe.\\ % sowie einen Ausflug in die Medienwissenschaften.
Leider stehen die Termine für Medienwissenschaften noch nicht fest. Die Vorlesungen zu Einführung in die Medienwissenschaften und User Interface Design müssten aber in den kommenden Wochen erscheinen. Bitte haltet Euch diesbezüglich auf dem Laufenden.\\
\fcolorbox{red}{white}{
		\begin{minipage}[t]{
			\textwidth}\textbf{Achtung!} Aufgrund der aktuellen Lage bezüglich COVID-19 können sich die Vorlesungstermine für dieses Semester noch ändern. Termine für Vorlesungen, die mit einem $^*$ markiert sind, sind unter Vorbehalt. Schaut am besten auf Alma (\url{https://alma.uni-tuebingen.de/alma/pages/cm/exa/coursecatalog/showCourseCatalog.xhtml?_flowId=showCourseCatalog-flow&_flowExecutionKey=e1s1}), ob die Termine dort geupdatet wurden.
		\end{minipage}}

\begin{minipage}{\textwidth}
    \footnotesize
\begin{center}
	\begin{tabular}{|c|c|c|c|c|c|}
	\hline
	 Zeit     &    Montag                    & Dienstag          & Mittwoch          & Donnerstag & Freitag \\ \hline\hline
	 08 -- 09 &    Mathematik I              &                   & Mathematik I      &  &  \\ \cline{1-1} \cline{3-3} \cline{5-6} 
	 09 -- 10 &    (\Matheprof), TBA         &                   & (\Matheprof), TBA &  &  \\ \hline
	 10 -- 11 &                              &                   &                   &  &  \\ \cline{1-1} \cline{3-6} 
	 11 -- 12 &                              &                   &                   &  &  \\ \hline
	 12 -- 13 &                              &                   &                   &  &  \\ \hline
	 13 -- 14 &                              &                   &                   &  &  \\ \hline
	 14 -- 15 &                              & Informatik I$^{*1}$  &                   &  &  \\ \cline{1-2} \cline{4-6} 
	 15 -- 16 &                              & (\Infoprof), TBA  &                   &  &  \\ \hline
	 16 -- 17 &                              &                   &                   &  &  \\ \hline
	 17 -- 18 &                              &                   &                   &  &  \\ \hline 
	\end{tabular}

%\scriptsize HS 22 = Kupferbau, Hörsaal 21 \\
%1: Erste Vorlesung am 21.10.\\
%2: \textbf{Unter Vorbehalt}, sofern eine Vertretungsprofessur der Medieninformatik zustande kommt. Raum und Uhrzeit können sich noch ändern, bitte schaut vor Beginn der Vorlesungen nochmal auf CAMPUS nach!
\\
1: Vorlesung findet dieses Semester asynchron, also online in Videoformat statt. (Link folgt) Übungen werden sowohl online als auch präsent angeboten.\\
\\

\end{center}
\end{minipage}

Dieser Plan gilt für das erste Semester Medieninformatik. 
Es kommen noch einige Übungsstunden zu den einzelnen Vorlesungen dazu. Die Zeiten für die Übungsgruppen werden jeweils in der ersten Vorlesung bekannt gegeben.
Bitte beachtet, dass die Vorlesung "`Einführung in die Medienwissenschaften"' erst \textbf{in der zweiten Vorlesungswoche} beginnt.
%Außerdem könnt ihr noch Veranstaltungen aus dem Bereich "`Schlüsselqualifikationen"' belegen.\\
Ab dem dritten Semester belegt ihr zusätzliche Kurse entsprechend eurer Profilwahl. Dazu gibt es aber im Vorfeld noch eine Infoveranstaltung.\\
Unter \texttt{http://www.medieninformatik.uni-tuebingen.de/} erreicht ihr die Website für den Fachbereich Medieninformatik mit allen wichtigen Infos für euer Studium. 
%Besonders interessant: In der ersten Vorlesungswoche wird es ein Treffen für die neuen Medieninformatiker geben, bei dem ihr euch untereinander und auch die Professoren kennen lernen könnt und ein paar Tipps zum Studium bekommt. Der Termin wird auf der Website bekannt gegeben. %und beim Anfängerfrühstück der Fachschaft % ist das so noch aktuell?
