Wie du dem folgenden Stundenplan entnehmen kannst, enthält das Lehramtsstudium der Informatik im ersten Semester neben der Informatik I-Vorlesung noch einen Ausflug in die technische Informatik. 
Beachte, dass man auch pädagogische Studien schon im ersten Semester besuchen kann. Falls sich Veranstaltungen aus dem zweiten
Fach mit Informatik überschneiden, kann man z.B. die Mathematik I-Vorlesung (eigentlich im dritten Semester vorgesehen) vorziehen. Achtung: Falls du \textbf{im zweiten Fach Mathematik} studierst: Die Inhalte der Mathematik 1 für Informatiker werden bereits durch andere Fächer aus der Mathematik abgedeckt, deshalb muss Mathematik 1 für Informatiker nicht belegt werden. Stattdessen gibt es ein sog. Ausgleichsmodul, welches für eine andere Prüfungsleistung mit gleichem Aufwand steht. Als Ausgleichsmodul kannst du jedes Wahlpflichtmodul mit gleicher Anzahl an LP wählen, welches in Campus angeboten wird. Dieses ist für das 3. Semester vorgesehen. \\
Wenn du im zweiten Fach Mathematik studierst, wird Lineare Algebra 1, Analysis 1 und Informatik 1 im ersten Semester empfohlen\footnote{Einführung in die technische Informatik dafür im dritten Semester}.\\
\\
\noindent\makebox[\textwidth][c]{%
	\setlength{\fboxrule}{4pt}
	\fcolorbox{red}{white}{
		\begin{minipage}[t]{
			%\textwidth}\textbf{Achtung!} Aufgrund der aktuellen Lage bezüglich COVID-19 können sich die Vorlesungstermine für dieses Semester noch ändern. Schau am besten auf Alma (\url{https://alma.uni-tuebingen.de/}), ob die Termine dort geupdatet wurden.
			\textwidth}\textbf{Achtung!} Die Daten für die Vorlesungstermine können sich noch ändern. Schau am besten auf Alma (\url{https://alma.uni-tuebingen.de/}), ob die Termine dort geupdatet wurden.
		\end{minipage}}}

%, zum Beispiel "`Einführung in die Schulpädagogik"' (Mi., 8-10, Kupferbau HS 25).

\begin{minipage}{\textwidth}
    \footnotesize
\begin{center}
	\begin{tabular}{|c|c|c|c|c|c|}
	\hline
Zeit     & Montag                    	& Dienstag          & Mittwoch          	& Donnerstag 	& Freitag \\ \hline\hline
08 -- 09 &    				            &                   &				     	&  			 	&  \\ \hline
09 -- 10 &    					        &                   & 						&  			 	&  \\ \hline
10 -- 11 & Techn. Informatik I (TI I)	&                   &                   	&  			 	&  \\ \cline{1-1} \cline{3-6}
11 -- 12 & (Prof. Bringmann)			&                   &                   	& 			 	&  \\ \hline
12 -- 13 &                              &                   &  						& 			 	&  \\ \cline{1-3} \cline{3-6}
13 -- 14 &                              &                   & TI I (Prof. Bringmann)&  			 	&  \\ \hline
14 -- 15 &                              & Informatik I  	&                 		&  Informatik I &  \\ \cline{1-2} \cline{4-4} \cline{6-6} 
15 -- 16 &                              & (\Infoprof)  		&                   	&  (\Infoprof) 	&  \\ \hline
16 -- 17 &                              &                   &                   	&  				&  \\ \hline
17 -- 18 &                              &                   &                   	&  				&  \\ \hline
	\end{tabular}

\end{center}
\end{minipage}

Dieser Plan gilt für das erste Semester Informatik im Bachelor of Education. Es kommen noch jeweils zwei Übungen zu den Vorlesungen dazu.
Die Zeiten für die Übungsgruppen werden innerhalb der ersten Woche in den Vorlesungen bekannt gegeben. Wenn du dir noch einen genaueren Überblick über den Studienverlauf verschaffen möchtest, dann schau doch mal ins Modulhandbuch oder schreib eine Mail an \texttt{lehramt@informatik.uni-tuebingen.de}.

