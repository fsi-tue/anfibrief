\ifml

\fett{Public transport in Tübingen}
The best way to get to the "`Morgenstelle"' (see city map on the last page) is to take the lines 5, 13, 14, 17, 18, X15 and 19 of the Tübinger
City bus (destination stop BG-Unfallklinik\footnote{\textbf{not} "`Auf der Morgenstelle"'!}). There is parking lot at the Morgenstelle,
for their use, however, one has to apply for an activation of the identity card, furthermore, the number of parking spaces is very limited and one often doesn't get any more space here from about 10 o'clock.

\ifmaster
Most Master events/lectures take place on the "`Sand"', the place of the WSI. To get to the Sand, take line 2, stop \emph{Sand Drosselweg}. Then about 200 meters further south and at the end of the street through the archway. Behind the main building Sand 14 there is a large parking lot, so the Sand is easy to reach by car. The large lecture halls are in building Sand 6, the seminar rooms in the main building Sand 1 and Sand 14. Some Machine Learning lectures take place in the
Maria-von-Linden-Straße 6. To get there, take line 3, stop \emph{Maria-von-Linden-Straße}, then head towards the observatory.
\fi
With your semester fee you have already paid part of the semester ticket for the bus (in addition to the student union fee and the matriculation fee).
The ticket costs \ticketpreis~EUR and is valid in the whole Naldo area, a public transport system around
Tübingen (unfortunately not to Stuttgart, but to Überlingen at Lake Constance). The ticket is valid from 1 October and is available in the DB travel centres and online\footnote{\url{https://tickets.naldo.de/}}. In order to buy the semester ticket offline, you will need to show your student ID and the Semester Ticket Certificate from the data control sheet. For the online purchase you need a login ID of the university (starting with \texttt{zx}).\\
Since the winter semester 2014/2015 Naldo's leisure time regulations have also been in force: Here you can use buses and trams within the Naldo area free of charge on weekdays from 7 p.m. and all day on weekends. All you need is your student card with the Naldo logo on it.
Especially interesting if the parents are visiting Tübingen: On Saturdays you can take a free bus in the city area (tariff level 11)\footnote{At least at the time of the editorial deadline this regulation still applies}, no matter if you are studying or not.\\ \\
An electronic bus schedule can be found online\footnote{\url{https://naldo.de/}} and in the Naldo app for Android and IOS.

\fett{Housing in Tübingen}
Unfortunately, the topic of housing has not lost its explosive power in Tübingen in recent years,
so that it may be difficult to find accommodation. We recommend the following two Studierendenwerke as your first points of contact:
Application forms for a place in a hall of residence can be obtained from either the Studierendenwerk AdöR in the administration of the dormitory,
Fichtenweg 5, 72076 Tübingen, or at the Studierendenwerk e.V., Rümelinstraße 8, 72070 Tübingen. For the private housing market, Wednesday and Saturday
editions of the Schwäbisches Tagblatt are available.
On the Internet, \url{www.wg-gesucht.de}, \url{www.zwischenmiete.de} and \url{www.studenten-wg.de} established.
You can find more on of our homepage at \url{https://www.fsi.uni-tuebingen.de/infos/anfi-faq\#wohnen}.
\else

\fett{Verkehr in Tübingen}
Zur Morgenstelle (siehe Stadtplan auf der letzten Seite) kommt ihr am besten mit den Linien 5, 13, 14, 17, 18, X15 und 19 des Tübinger
Stadtbusses (Ziel-Haltestelle BG-Unfallklinik\footnote{\textbf{nicht} "`Auf der Morgenstelle"'!}). Es gibt zwar auch Parkplätze auf der Morgenstelle,
für deren Benutzung muss man jedoch eine Freischaltung des Ausweises beantragen, außerdem ist die Anzahl an Parkplätzen sehr begrenzt und man bekommt hier oft ab ca. 10 Uhr keinen Platz mehr.
\ifmaster
Die meisten Master-Veranstaltungen finden auf dem Sand, dem Sitz des WSI, statt. Den Sand erreichst du mit der Linie 2, Haltestelle \emph{Sand Drosselweg}. Dann ca. 200 Meter weiter Richtung Süden und am Ende der Straße durch den Torbogen. Hinter dem Hauptgebäude Sand 14 befindet sich ein großer Parkplatz, der Sand ist also auch mit dem Auto gut zu erreichen. Die großen Hörsäle befinden sich im Gebäude Sand 6, die Seminarräume im Hauptgebäude Sand 1 bzw. Sand 14.
\fi
Mit Eurem Semesterbeitrag habt ihr (neben dem Studierendenwerksbeitrag und der Immatrikulationsgebühr) schon einen Teil des Semester-Tickets für den Bus bezahlt.
Das Ticket kostet \ticketpreis~EUR und gilt im ganzen Naldogebiet, einem Verkehrsverbund rund um
Tübingen (leider nicht bis Stuttgart, dafür bis Überlingen am Bodensee). Das Ticket gilt ab 1. Oktober und ist u. a. beim Verkehrsverein
(Neckarbrücke), in den Reisezentren der DB und online\footnote{\url{https://tickets.naldo.de/}} erhältlich. Um das Semesterticket offline kaufen zu können, müsst ihr euren Studiausweis sowie die "`Bescheinigung für das Semesterticket"' vom Datenkontrollblatt vorzeigen. Für den Online-Kauf benötigt ihr eine Login-ID der Uni (beginnend mit \texttt{zx}).\\
Seit dem Wintersemester 2014/2015 gilt außerdem die Freizeitregelung von Naldo: Hier könnt ihr unter der Woche ab 19 Uhr und am Wochenende ganztägig Busse und Bahnen innerhalb des Naldo-Gebiets kostenlos nutzen. Dafür braucht ihr einfach nur euren Studiausweis, auf dem sich das Naldo-Logo befindet.
Besonders interessant, falls die Eltern zu Besuch in Tübingen sind: Samstags kann man im Stadtgebiet (Tarifstufe 11) kostenlos Bus fahren\footnote{Zumindest zum Zeitpunkt des Redaktionsschlusses gilt diese Regelung noch}, egal ob man studiert oder nicht. \\\\
Eine elektronische Fahrplanauskunft findet sich online\footnote{\url{https://naldo.de/}} und in der Naldo-App für Android und IOS.

\fett{Wohnen in Tübingen}
Das Thema Wohnen hat in Tübingen in den letzten Jahren leider nicht an Brisanz verloren, so dass
es unter Umständen schwierig ist, eine Unterkunft zu finden. Als erste Anlaufstellen empfehlen wir
die beiden Studierendenwerke: Antragsformulare für einen Wohnheimplatz gibt es entweder beim
Studierendenwerk AdöR in der Wohnheimverwaltung, Fichtenweg 5, 72076 Tübingen, oder beim Studierendenwerk
e.V., Rümelinstraße 8, 72070 Tübingen. Für den privaten Wohnungsmarkt sind Mittwochs- und Samstagsausgabe
des Schwäbischen Tagblatts zu empfehlen. Weiterhin ist auch die Studentische Zimmervermittlung in der Mensa
Wilhelmstraße sehr hilfreich. Im Internet haben sich \url{www.wg-gesucht.de}, \url{www.zwischenmiete.de} und
\url{www.studenten-wg.de} etabliert. Mehr gibt es auf
unserer Homepage unter \url{https://www.fsi.uni-tuebingen.de/infos/anfi-faq\#wohnen}.
\fi
