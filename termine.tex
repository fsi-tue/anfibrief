\begin{description}

\ifkogwiss
\ifmaster
\item[Montag, 1. Oktober \Jahr, 10 Uhr, Alte Physik, ÜR08]\ \\
Heute beginnt der Vorbereitungskurs Mathematik speziell für Kognitionswissenschaftler im Master. Es ist nicht Pflicht daran teilzunehmen, es ist aber sehr empfehlenswert. Nicht zuletzt lernt ihr hier erste Freunde kennen! Der Vorkurs bietet euch eine Zusammenfassung des Mathestoffs, der im Bachelor Kognitionswissenschaft an der Uni Tübingen behandelt wird.
\textbf{Wenn ihr euren Bachelor nicht an der Universität Tübingen oder in einem anderen Fach erworben habt, kann der Vorkurs für euch sinnvoll sein.} Es wäre nett, wenn ihr euch mit einer kurzen Mail an \texttt{patrick.weigert \At student.uni-tuebingen.de} anmelden würdet.
Stattfinden wird der Vorkurs im ÜR 08 in der alten Physik, das ist die Gmelinstraße 6. Diese befindet sich an der Bushaltestelle Gmelinstraße, nördlich gegenüber von der Neuen Aula. Wenn ihr auf der Wilhelmstraße seid, geht rechts an der Neuen Aula vorbei, dort findet ihr die Alte Physik an der Ecke Gmelin- und Nauklerstraße, rechts von der Neuen Aula. Seid ihr auf der Hölderlinstraße, geht links dran vorbei und die Alte Physik ist auf der linken Straßenseite. Genaue Informationen bezüglich Treffpunkt am ersten Termin erhaltet ihr dann nochmal per Mail.

\seticon{faBus}~\textbf{Bushaltestelle:} Gmelinstraße, Hölderlinstraße, Uni/Neue Aula

\else
\item[Montag, 1. Oktober \Jahr, 10 Uhr, Sand 6, Raum F119]\ \\
Heute beginnt der Vorbereitungskurs Mathematik. Es ist nicht Pflicht daran teilzunehmen,
es ist aber sehr empfehlenswert. Nicht zuletzt lernt ihr hier erste Freunde kennen!
\ifsommersemester
Der Vorkurs bietet dir eine Wiederholung des Schulstoffes sowie eine Übersicht über den Stoff von Mathe II
\fi
\ifwintersemester
Der Vorkurs bietet dir eine Wiederholung des Schulstoffes sowie eine Übersicht über den Stoff von Mathe I
\fi
und führt euch in die Terminologie ein, die ihr in den Mathe-Vorlesungen wiederfinden werdet.
\ifmaster
\\
\textbf{Wenn ihr euren Bachelor nicht an der Universität Tübingen oder in einem anderen Fach erworben habt, kann der Vorkurs für euch sinnvoll sein. Auf unserer Website findet ihr ein Skript, von dem ein Teil auch im Vorkurs besprochen wird. Damit solltet ihr einschätzen können, wie viel vom Stoff bereits bekannt ist und ob sich der Besuch des Vorkurses lohnt.}
\fi
Um am Vorkurs teilzunehmen, müsst ihr euch bis zum 27.09. \Jahr~anmelden. Weitere Informationen zur Anmeldung findet ihr in Kürze auf unserer Webseite \url{https://www.fsi.uni-tuebingen.de/erstsemester/veranstaltungen/}.
% Um Anmeldung wird gebeten, sie ist aber nicht zwingend erforderlich.

\ifsommersemester
Bitte beachtet, dass das Semesterticket \emph{erst ab dem 1.4} gültig ist. Für die Tage davor müsst ihr also extra Tickets kaufen. Natürlich kommt man zur Morgenstelle auch zu Fuß oder mit dem Rad, da es jedoch steil den Berg hochgeht, muss man hierfür einiges an Zeit einrechnen.
\fi
Der Vorkurs findet auf dem Gelände des Wilhelm-Schickard-Instituts (oft einfach nur \emph{der Sand} genannt) statt. Ihr erreicht den Sand per Bus mit der Linie 2, Haltestelle Sand Drosselweg. Dann ca. 200 Meter Richtung Süden und durch den großen gelben Torbogen. Per Auto erreicht ihr den Sand über den Nordring (folgt den Schildern "`Uni-Sand"'), hinter dem Hauptgebäude Sand 14 befindet sich ein großer Parkplatz. Der Hörsaal F119 befindet sich jedoch nicht im Hauptgebäude, sondern im Gebäude Sand 6. Nach dem Torbogen direkt links und an den Bäumen entlang, dann steht ihr vor dem Haupteingang von Sand 6. Der Hörsaal befindet sich hinter der ersten Tür links.

\seticon{faBus}~\textbf{Bushaltestelle:} Linie 2, Sand Drosselweg (Rest ausgeschildert)
\fi

%TODO Hier Master Kogni-Vorkurs

\else
\item[Montag, 1. Oktober \Jahr, 10 Uhr, Sand 6, Raum F119]\ \\
Heute beginnt der Vorbereitungskurs Mathematik. Es ist nicht Pflicht daran teilzunehmen,
es ist aber sehr empfehlenswert. Nicht zuletzt lernt ihr hier erste Freunde kennen!
\ifsommersemester
Der Vorkurs bietet dir eine Wiederholung des Schulstoffes sowie eine Übersicht über den Stoff von Mathe II
\fi
\ifwintersemester
Der Vorkurs bietet dir eine Wiederholung des Schulstoffes sowie eine Übersicht über den Stoff von Mathe I
\fi
und führt euch in die Terminologie ein, die ihr in den Mathe-Vorlesungen wiederfinden werdet.
\ifmaster
\\
\textbf{Wenn ihr euren Bachelor nicht an der Universität Tübingen oder in einem anderen Fach erworben habt, kann der Vorkurs für euch sinnvoll sein. Auf unserer Website findet ihr ein Skript, von dem ein Teil auch im Vorkurs besprochen wird. Damit solltet ihr einschätzen können, wie viel vom Stoff bereits bekannt ist und ob sich der Besuch des Vorkurses lohnt.}
\fi
Um am Vorkurs teilzunehmen, müsst ihr euch bis zum 27.09. \Jahr~anmelden. Weitere Informationen zur Anmeldung findet ihr in Kürze auf unserer Webseite \url{https://www.fsi.uni-tuebingen.de/erstsemester/veranstaltungen/}.
% Um Anmeldung wird gebeten, sie ist aber nicht zwingend erforderlich.

\ifsommersemester
Bitte beachtet, dass das Semesterticket \emph{erst ab dem 1.4} gültig ist. Für die Tage davor müsst ihr also extra Tickets kaufen. Natürlich kommt man zur Morgenstelle auch zu Fuß oder mit dem Rad, da es jedoch steil den Berg hochgeht, muss man hierfür einiges an Zeit einrechnen.
\fi
Der Vorkurs findet auf dem Gelände des Wilhelm-Schickard-Instituts (oft einfach nur \emph{der Sand} genannt) statt. Ihr erreicht den Sand per Bus mit der Linie 2, Haltestelle Sand Drosselweg. Dann ca. 200 Meter Richtung Süden und durch den großen gelben Torbogen. Per Auto erreicht ihr den Sand über den Nordring (folgt den Schildern "`Uni-Sand"'), hinter dem Hauptgebäude Sand 14 befindet sich ein großer Parkplatz. Der Hörsaal F119 befindet sich jedoch nicht im Hauptgebäude, sondern im Gebäude Sand 6. Nach dem Torbogen direkt links und an den Bäumen entlang, dann steht ihr vor dem Haupteingang von Sand 6. Der Hörsaal befindet sich hinter der ersten Tür links.

\seticon{faBus}~\textbf{Bushaltestelle:} Linie 2, Sand Drosselweg (Rest ausgeschildert)
\fi 

\item[Dienstag, 2. Oktober \Jahr, 19:30 Uhr, Sand, Raum A301 (Treffpunkt ausgeschildert)]\ \\
Am Abend des zweiten Vorkurstages möchten wir euch zu einem gemütlichen Filmabend auf dem Sand einladen.
Hier habt ihr die Möglichkeit, bei einem Film \footnote{welcher Film gezeigt wird, dürfen wir aus lizenzrechtlichen Gründen nicht bekannt geben} zu entspannen, einige Fachschaftler, den Sand und eure zukünftigen Kommilitonen kennen zu lernen.

\seticon{faBus}~\textbf{Bushaltestelle:} Linie 2, Sand Drosselweg (Rest ausgeschildert)

\item[Mittwoch, 3. Oktober \Jahr~(weitere Infos folgen)]\ \\
Das Wandern ist des Infos Lust - unter Leitung eines Fachschaftlers könnt ihr bei hoffentlich schönem Wetter eine Wanderung durch den Schönbuch (ca. 15km) unternehmen. \\
Weitere Infos wie Uhrzeit und Treffpunkt finden sich in Kürze auf \url{https://www.fsi.uni-tuebingen.de/erstsemester/veranstaltungen/}.

\item[Donnerstag, 4. Oktober \Jahr, Sand, Raum A301 (Uhrzeit folgt)]\ \\
Wir möchten euch zu einem kleinen (analog-) Spieleabend mit guter Gesellschaft und entspannter Atmosphäre auf dem Sand einladen. Für einige Spiele sowie Getränke und Knabberkram (gegen einen kleinen Obolus) sorgt die Fachschaft. Wir freuen uns natürlich sehr, wenn ihr auch eigene Spiele mitbringt! Details werden noch auf \url{https://www.fsi.uni-tuebingen.de/erstsemester/veranstaltungen/} bekannt gegeben.

\seticon{faBus}~\textbf{Bushaltestelle:} Sand Drosselweg (Rest ausgeschildert)

\ifmaster
\ifbinfo
\item[Donnerstag, 4. Oktober \Jahr, 9 Uhr, Sand]\ \\
 Heute beginnt ein Informatik-Vorkurs speziell für Bioinformatik-Studenten im Master. Dieser Vorkurs wird dringend empfohlen, wenn du aus einem fachfremden Studiengang wie z.B. Biologie oder anderen Lebenswissenschaften kommst und noch keine oder sehr wenig Erfahrung in der Informatik und der Programmierung (CLI, Java, Python, \LaTeX) hast. Der Vorkurs wird in Englisch gehalten. Alle weitere Informationen und die Anmeldung findest du auf folgender Website: \\ \url{http://www.wsi.uni-tuebingen.de/studium/infos-fuer-anfaenger-und-studieninteressierte/vorkurs-informatik-fuer-biologen.html}

\seticon{faBus}~\textbf{Bushaltestelle:} Linie 2, Sand Drosselweg
\fi
\fi


\item[Freitag, 5. Oktober \Jahr, 20 Uhr, Neckarmüller]\ \\
Nachdem sich der Stress der ersten Vorkurswoche gelegt hat, möchten wir mit euch feierlich das Wochenende einläuten.
Dazu treffen wir uns um 20 Uhr bei der Neckarbrücke (\emph{vor} dem Gasthaus "`Neckarmüller"') zu einer ausgiebigen Kneipentour.

\seticon{faBus}~\textbf{Bushaltestelle:} Neckarbrücke


\ifkogwiss

\item[Montag, 8. Oktober \Jahr ]\ \\
Damit sich die Kognis untereinander kennenlernen, gibt es einen Abend nur für diese. Hier soll auf dem Sand ab 17 Uhr Pizza gebacken und  Spiele, wie z.B. Volleyball oder Kicker, gespielt werden.
Treffpunkt ist auf der Terasse des Sandes.
\fi

\item[Dienstag, 9. Oktober \Jahr, 19 Uhr, Neckarmüller]\ \\
An diesem Abend wirst du in Form einer Nachtrallye Tübingen erkunden können. Du wirst die schönsten Ecken der Stadt sehen, Historisches wie Nützliches hören, dabei hoffentlich die Orientierung in deiner neuen Heimat etwas verbessern und kannst Kontakte knüpfen. Der Treffpunkt ist bei der Neckarbrücke (\emph{vor} dem Gasthaus „Neckarmüller“).

\seticon{faBus}~\textbf{Bushaltestelle:} Neckarbrücke

\item[Mittwoch, 10. Oktober \Jahr, 17 Uhr, Sand (Grillstelle)]\ \\
Wir laden euch an diesem Nachmittag am vorletzten Tag des Mathe-Vorkurses zu
einem Treffen auf dem Sand ein, bei dem ihr euch in gemütlicher Runde mit
anderen Studierenden höherer Semester austauschen könnt. Wir bieten euch
Gesellschaft und einen heißen Grill. Grillgut, Getränke, Geschirr und Besteck solltet ihr zur unkomplizierteren Organisation jedoch selbst mitbringen.
Falls sich die Uhrzeit noch ändern sollte, werden wir dies auf \\ \url{https://www.fsi.uni-tuebingen.de/erstsemester/veranstaltungen/} bekannt geben.

\seticon{faBus}~\textbf{Bushaltestelle:} Sand Drosselweg (Rest ausgeschildert)

\item[Donnerstag, 11. Oktober \Jahr, 18 Uhr, Sand]\ \\
An diesem Abend möchten wir euch die Gelegenheit bieten, eure Laptops für den Uni-Alltag einzurichten. Neben Installation von diversen Toolkits und Entwicklungsumgebungen soll hierbei soll auch die Gelegenheit nicht ausbleiben, in alten Erinnerungen zu schwelgen\footnote{\url{https://xkcd.com/422/}}.

\seticon{faBus}~\textbf{Bushaltestelle:} Sand Drosselweg (Rest ausgeschildert)


\item[Freitag, 12. Oktober \Jahr, 9 Uhr, Mensa Morgenstelle (Treffpunkt ausgeschildert)]\ \\
Wir laden euch an diesem Morgen zu einem gemütlichen Frühstück ein! Dabei erfahrt ihr einiges über die Uni, die Fachschaft und was euch in den nächsten Monaten erwartet -- auch im Gespräch mit älteren
Studierenden. Außerdem werdet ihr durch Prof. \Infoprof~-- er wird die Informatik I Vorlesung halten -- begrüßt.
\ifmaster Zwar ist Informatik I eine Bachelor-Veranstaltung, aber ihr werdet Prof. \Infoprof~ vielleicht auch in Master-Vorlesungen kennen lernen. \fi
Danach machen wir eine Führung über die Morgenstelle, damit ihr die wichtigsten Räume und Hörsäle kennen lernt.

Zur Morgenstelle kommt man entweder mit dem Bus, zu Fuß oder mit dem Rad. Da der Weg zur Morgenstelle aber sehr steil ist (Tübingen ist hügelig), sollte man hierfür einiges an Zeit einrechnen.
Der einfachste Weg ist hier über das Parkhaus König. Von dort müsst ihr bergauf der Schnarrenbergstraße folgen. Es geht dann zunächst an den Uni-Kliniken Berg vorbei, anschließend erreicht ihr die Morgenstelle. Falls ihr aus der Richtung Waldh\"auser-Ost kommt, so m\"usst ihr dem Nordring in Richtung Kliniken Berg folgen. Für beide Wege solltet ihr jeweils mindestens eine halbe Stunde zu Fuß einrechnen.
Wenn ihr mit dem Auto ankommt k\"onnt ihr (kostenpflichtig) an den Straßenseiten des Nordrings, oder im Parkhaus "`Ebenhalde"' oberhalb der Morgenstelle parken. Am einfachsten geht es jedoch mit dem Bus.
%Wenn du Lust hast, kannst du ab 11:45 Uhr das Mensaessen ausprobieren.

\seticon{faBus}~\textbf{Bushaltestelle:} BG Unfallklinik (Linie 5, 13, 14, 17, 18, 19, X15)

\iflehramt
\item[Freitag, 12. Oktober \Jahr, 10-12 Uhr, Kupferbau, Hörsaal 25]\ \\
Studienanfänger/innen im Lehramtsstudium können zum Frühstück nicht lange bleiben, denn bereits um 10 Uhr werdet ihr über die speziellen
Anforderungen des Lehramtsstudiums informiert. Neben den eigentlichen Fachinhalten kommen im
Bachelor of Education noch einige andere Dinge auf euch zu, z.B. ein Orientierungspraktikum, die
Fachdidaktik und der Studienbereich Bildungswissenschaften. Zu Beginn des Masters steht dann
ein Schulpraxissemester an, und nach dem Studienabschluss das Referendariat. Bei dieser Veranstaltung werden euch
alle diese Elemente des Lehramtsstudiums vorgestellt.

\seticon{faBus}~\textbf{Bushaltestelle:} Hölderlinstraße bzw. Uni/Neue Aula
\fi

% Diese if-else-Konstruktion ist Kunst! -- Tim
\item[Freitag, 12. Oktober \Jahr, 13{\ifkogwiss}:45{\fi} Uhr, Sand (Räume und Programm folgen)]\ \\
Um 13{\ifkogwiss}:45{\fi} Uhr treffen wir uns auf dem „Sand“ (dem Sitz des Wilhelm-Schickard-Instituts)
wieder. Hier werden wir euch mit Informationen rund um{\ifkogwiss} {\else} euer Studium und {\fi}die Arbeit unserer Lehrstühle auf dem Sand versorgen.\\

{\ifkogwiss}Als Studierende der Kognitionswissenschaften könnt ihr um 13:45 Uhr zusto{\ss}en, die anderen Studiengänge erhalten vorab eine spezifische Einführung, die für euch Montag, 15. Oktober am Psychologischen Institut stattfindet. {\else}{\ifinfo}Als Lehramt Informatik und Informatik-Studierende werden euch auch mögliche Nebenfächer vorgestellt, ihr werdet {\else}Ihr werdet als *-Informatik-Studierende {\fi}noch einmal spezifisch begrüßt und ihr erhaltet einen studiengangsspezifischen Einblick in Forschung und Lehre.{\fi}

{\ifkogwiss}Die {\else}Nachdem ihr über den Verlauf des Studiums der ersten Wochen, Monate und Semester informiert wurdet, bieten euch die {\fi}verschiedenen Fachbereiche {\ifkogwiss}auf dem Sand bieten euch {\else} {\fi}mit Vorträgen einen Einblick in ihre Arbeit. Die Fachbereiche kennen zu lernen lohnt sich für Studierende aller Studiengänge, die Vorträge zeigen welche Themen am Wilhelm-Schickard-Institut verfolgt werden, geben euch ein Gefühl, was für euch hier interessant sein kann und wo ihr möglicherweise Schwerpunkte im Studium, einer Studien- oder Abschlussarbeit setzten wollt. Je nach Interesse könnt ihr euch verschiedene Vorträge anhören, allerdings bedarf dies an Auswahl und Planung von eurer Seite. Die angebotenen Vorträge können sich thematisch oder, wie die Begrüßungen, im Zeitrahmen ändern. Schaut daher auch kurzfristig (am selben Tag) auf \url{https://www.fsi.uni-tuebingen.de/erstsemester/veranstaltungen/} nach.

\seticon{faBus}~\textbf{Bushaltestelle:} Sand Drosselweg (Rest ausgeschildert)

% Master-Kaffeerunde
\ifmaster
\item[Freitag, 12. Oktober \Jahr, 16 Uhr, Sand (Raum folgt)]\ \\
Am Nachmittag des Anfi-Frühstücks und im Anschluss an die Lehrstuhlvorstellung wird speziell für neue Masterstudenten ein lockerers Treffen mit Kaffee, Keksen und vielen Informationen zu eurem Studium angeboten.
Hier habt ihr die Möglichkeit, Fragen zu stellen, letzte Fragen zu eurem Stundenplan zu klären und schon mal eure Kommilitonen zu treffen, die ihr in den Vorlesungen wiederfinden werdet. \\
%   Die Uhrzeit und der genaue Ort können sich noch ändern, schau daher vorher nochmal auf \url{https://www.fsi.uni-tuebingen.de/erstsemester/veranstaltungen/} nach.
Der genaue Ort wird noch festgelegt, schau daher vorher nochmal auf \url{https://www.fsi.uni-tuebingen.de/erstsemester/veranstaltungen/} nach.
\fi

% erste Vorlesung
\ifbachelor
\item[Montag, 15. Oktober \Jahr, Morgenstelle, Hörsaal N7]\ \\
Deine erste Vorlesung beginnt um
\ifwintersemester 8 Uhr -- Ihr habt „Mathe I für Informatiker“  \fi
\ifsommersemester 10 Uhr -- Ihr habt „Mathe II für Informatiker“  \fi
bei Prof. \Matheprof.
%bei Dr. \Matheprof.
Alles, was ihr heute (und in Zukunft) benötigt: persönlichen Wachmacher, einen Stift, einen Block und den Studierendenausweis.

\seticon{faBus}~\textbf{Bushaltestelle:} BG Unfallklinik
\fi




%TODO @Paul: Kogni-Krams

\ifkogwiss
\ifbachelor
\item[Montag, 15. Oktober \Jahr, 17 Uhr, Psychologisches Institut, Hörsaal]\ \\
An diesem Abend werdet ihr von Frau Prof. Rolke und Frau Jendreyko begrüßt. Zudem stellen sich euch, nach einer kurzen Einführung in den Studienaufbau, die Lehrstühle der Kognitionswissenschaft mit ihren Themen und Forschungsgebieten vor.
\fi
%\ifmaster
%\item[Montag, 15. Oktober \Jahr, 15:00 Uhr, Psychologisches Institut, Seminarraum 4.326 ]\ \\
%Heute Abend stellen sich euch die verschiedene Lehrstühle der Kognitionswissenschaft vor, um einen Einblick in ihre Forschungsthemen zu ermöglichen.
%\fi

\seticon{faBus}~\textbf{Bushaltestelle:} Hölderlinstraße / Uni-Kliniken Tal
\fi

\ifkogwiss
\ifmaster
\item[Montag, 15. Oktober \Jahr, 15:00 Uhr, Psychologisches Institut, Seminarraum 4.326 ]\ \\
Hier bekommen speziell Master-Studenten eine separate Begrüßung durch Frau Prof. Rolke, die euch einen ersten Einblick ins Studium gibt, indem sie den regulären Studienaufbau des Master Kognitionswissenschaft erläutert. %Ort und Zeit werden noch auf \url{https://www.fsi.uni-tuebingen.de/erstsemester/veranstaltungen/} bekannt gegeben.
\fi
\fi




%TODO erneuern
\item[Donnerstag, 18. Oktober \Jahr,  Sand (weitere Infos folgen)]\ \\
An diesem Donnerstag möchten wir euch zunächst auf den Sand einladen, um in gemütlicher Runde mit anderen Kommilitonen und Fachschaftlern Brett- und Gesellschaftsspiele zu spielen. Der genaue Zeitpunkt sowie der Raum stehen noch nicht fest, schaut dafür bitte ein paar Tage vorher auf \url{https://www.fsi.uni-tuebingen.de/erstsemester/veranstaltungen/} nach.\\ Sobald die Zeit an diesem Abend ausreichend fortgeschritten ist, möchten wir zusammen mit euch das Clubhausfest (im Clubhaus, Wilhelmstr. 30 direkt gegenüber der Neuen Aula) besuchen. Dieses findet während der Vorlesungszeit jeden Donnerstag statt. Das Besondere daran ist, dass es jede Woche von einer anderen Fachschaft oder studentischen Gruppe organisiert wird und dementsprechende Abwechslung bietet -- ein Besuch lohnt sich also immer. Außerdem ist der Eintritt frei, ihr benötigt lediglich euren Studentenausweis.

\seticon{faBus}~\textbf{Bushaltestelle:} Sand Drosselweg (Rest ausgeschildert)

% gibbet im WS18 nicht
%\item[Donnerstag, 17. Mai \Jahr, 21 Uhr]\ \\
%Das anfängliche Chaos des Vorlesungsbeginns hat sich gelegt und auch an diesem Donnerstag findet das Clubhausfest statt. Heute lohnt sich der Besuch jedoch ganz besonders, denn die Fachschaften Informatik, Kogni und Psychologie sind Gastgeber!\\
~\\

\end{description}
