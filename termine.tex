\begin{description}
	
% TODO
\item[Montag, 26. März \Jahr, 10 Uhr, Morgenstelle, Hörsaal N3]\ \\
Heute beginnt der Vorbereitungskurs Mathematik. Es ist nicht Pflicht daran teilzunehmen, 
es ist aber sehr empfehlenswert. Nicht zuletzt lernst du hier erste Freunde kennen!
Der Vorkurs bietet dir eine Wiederholung des Schulstoffes sowie eine Übersicht über den Stoff von Mathe II 
und führt dich in die Terminologie ein, die du in den Mathe-Vorlesungen wiederfinden wirst.
\ifmaster
\\
\textbf{Wenn du deinen Bachelor nicht an der Universität Tübingen oder in einem anderen Fach erworben hast, kann der Vorkurs für dich sinnvoll sein. Auf unserer Website findest du ein Skript von dem ein Teil auch im Vorkurs besprochen wird. Damit solltest du einschätzen können, wie viel von dem Stoff du kennst und ob der Vorkurs für dich sinnvoll ist.}
\fi

Um Anmeldung wird gebeten, sie ist aber nicht zwingend erforderlich. 
Weitere Informationen findest du in Kürze auf unserer Webseite \url{https://www.fsi.uni-tuebingen.de/erstsemester/veranstaltungen/}.

Bitte beachtet, dass das Semesterticket \emph{erst ab dem 1.4} gültig ist. Für die Tage davor müsst ihr also extra Tickets kaufen. Natürlich kommt man zur Morgenstelle auch zu Fuß oder mit dem Rad, da es jedoch steil den Berg hochgeht, muss man hierfür einiges an Zeit einrechnen.
Zur Morgenstelle kommt man entweder mit dem Bus, zu Fuß oder mit dem Rad. Da der Weg zur Morgenstelle aber sehr steil ist (Tübingen ist hügelig), sollte man hierfür einiges an Zeit einrechnen.
Der einfachste Weg ist hier über das Parkhaus König. Von dort müsst ihr bergauf der Schnarrenbergstraße folgen. Es geht dann zunächst an den Uni-Kliniken Berg vorbei, anschließend erreicht ihr die Morgenstelle. Falls ihr aus der Richtung Waldh\"auser-Ost kommt, so m\"usst ihr dem Nordring in Richtung Kliniken Berg folgen. Für beide Wege solltet ihr jeweils mindestens eine halbe Stunde zu Fuß einrechnen.
Wenn ihr mit dem Auto ankommt k\"onnt ihr (kostenpflichtig) an den Straßenseiten des Nordrings, oder im Parkhaus "`Ebenhalde"' oberhalb der Morgenstelle parken. 
   
\textbf{Bushaltestelle:} BG Unfallklinik (Linie 5, 13, 14, 18, 19, X15)


\ifmaster
\ifbinfo
\item[Mittwoch, 4. April \Jahr, 9 Uhr, Sand, Raum C215]\ \\
 Heute beginnt ein Informatik-Vorkurs speziell für Bioinformatik-Studenten im Master. Dieser Vorkurs wird dringend empfohlen, wenn du aus einem fachfremden Studiengang wie z.B. Biologie oder anderen Lebenswissenschaften kommst und noch keine oder sehr wenig Erfahrung in der Informatik und der Programmierung (Java, Python, \LaTeX) hast. Der Vorkurs wird in Englisch gehalten. Alle weitere Informationen und die Anmeldung findest du auf folgender Website: \\ \url{http://www.wsi.uni-tuebingen.de/studium/infos-fuer-anfaenger-und-studieninteressierte/vorkurs-informatik-fuer-biologen.html}

\textbf{Bushaltestelle:} Linie 2, Sand Drosselweg
\fi
\fi 

% TODO: Welcher Raum? Raum anpassen!
\item[Mittwoch, 04. April \Jahr, 19:30 Uhr, Sand 13 Raum A301 (Treffpunkt ausgeschildert)]\ \\
Wir laden dich an diesem Abend zu einem Filmabend auf dem Sand ein.
Hier hast du die Möglichkeit, in gemütlicher Atmosphäre einige Fachschaftler und deine zukünftigen Kommilitonen kennen zu lernen. Snacks inklusive.

\textbf{Bushaltestelle:} Linie 2, Sand Drosselweg (Rest ausgeschildert) 


\item[Mittwoch, 10. April \Jahr, 20 Uhr, Neckarmüller]\ \\
  An diesem Abend wirst du in Form einer Nachtralley Tübingen erkunden können. Du wirst die schönsten Ecken der Stadt sehen, Historisches wie Nützliches hören, dabei hoffentlich die Orientierung in deiner neuen Heimat etwas verbessern und kannst Kontakte knüpfen. Der Treffpunkt ist bei der Neckarbrücke (\emph{vor} dem Gasthaus „Neckarmüller“).
  
  \textbf{Bushaltestelle:} Neckarbrücke   


\item[Mittwoch, 11. April \Jahr, 17:00 Uhr, Sand (Grillstelle)]\ \\
Wir laden euch an diesem Nachmittag am vorletzten Tag des Mathe-Vorkurses zu
einem Treffen auf dem Sand ein, bei dem ihr euch in gemütlicher Runde mit
anderen Studierenden höherer Semester austauschen könnt. Wir bieten euch
Gesellschaft und einen heißen Grill.
Grillgut, Geschirr und Besteck solltet ihr zur unkomplizierteren Organisation jedoch selbst mitbringen.\\
Falls sich die Uhrzeit noch ändern sollte, werden wir dies auf \\ \url{https://www.fsi.uni-tuebingen.de/erstsemester/veranstaltungen/} bekannt geben.

\textbf{Bushaltestelle:} Sand Drosselweg (Rest ausgeschildert)


%TODO wann und wo?
%\item[Donnerstag, 12. Oktober \Jahr, 16 Uhr, Sand]\ \\
%Um 16 Uhr treffen wir uns auf dem „Sand“ (dem Sitz des Wilhelm-Schickard-Instituts) vor dem Haupteingang
%wieder. Hier werden wir dich mit Informationen rund ums Studium versorgen.
%Außerdem werden dir mögliche Nebenfächer vorgestellt, zusätzlich werden die
%Studierenden der spezielleren Studiengänge noch einmal gesondert von ihren
%Ansprechpartnern begrüßt.
% Dort werdet Ihr gegen 14 Uhr durch Professor \Infoprof, Euren Informatik-I-Dozenten, begrüßt. 
% Nachdem Ihr über den Verlauf des Studiums in den ersten Wochen, Monaten und Semestern informiert wurdet,
%Hier wird es ebenfalls eine Führung durch das Gebäude geben, bei der die verschiedenen Fachbereiche einen Einblick in ihre Arbeit geben. Je nach Studiengang kann sich die Uhrzeit geringfügig ändern, schau daher 
%nochmal auf \url{https://www.fsi.uni-tuebingen.de/erstsemester/veranstaltungen/} nach.
%  Außerdem werden den Informatikstudenten mögliche Nebenfächer vorgestellt, zusätzlich werden die Studierenden der spezielleren Studiengänge noch einmal gesondert von ihren Ansprechpartnern begrü{\ss}t.

%\textbf{Bushaltestelle:} Sand Drosselweg (Rest ausgeschildert)


%TODO Ort, vsl Sand
\item[Freitag, 13. April \Jahr, 10 Uhr, Sand (Treffpunkt ausgeschildert)]\ \\
   Wir laden dich an diesem Morgen zu einem gemütlichen Frühstück ein! Dabei erfährst du
   von uns einiges über die Uni und was dich in den nächsten Monaten erwartet -- auch im Gespräch mit älteren
   Studierenden. Außerdem wirst du durch Prof. \Infoprof~-- er wird die Informatik I Vorlesung halten -- begrüßt.
   \ifmaster Zwar ist Informatik I eine Bachelor-Veranstaltung, aber du wirst Prof. \Infoprof~ vielleicht auch in Master-Vorlesungen kennen lernen. \fi 
   fällt weg, wird verschoben
   %Danach machen wir eine Führung über die Morgenstelle, damit du die wichtigsten Räume und Hörsäle kennen lernst. 
   %Wenn du Lust hast, kannst du ab 11:45 Uhr das Mensaessen ausprobieren.

   \textbf{Bushaltestelle:} BG Unfallklinik
   
\iflehramt
\item[Freitag, 11. April \Jahr, 14-16 Uhr, Kupferbau, Hörsaal 25]\ \\
Studienanfänger/innen im Lehramtsstudium werden hier gleich zu Beginn über die speziellen 
Anforderungen dieses Studiums informiert. Neben den eigentlichen Fachinhalten kommen im 
Bachelor of Education noch einige andere Dinge auf dich zu, z.B. ein Orientierungspraktikum, die 
Fachdidaktik und der Studienbereich Bildungswissenschaften. Zu Beginn des Masters steht dann 
ein Schulpraxissemester an, und nach dem Studienabschluss das Referendariat. Wir werden dir 
alle diese Elemente des Lehramtsstudiums vorstellen und versuchen, deine Fragen zu beantworten.

\textbf{Bushaltestelle:} Hölderlinstraße bzw. Uni/Neue Aula

\fi


%fällt weg da SS
%\ifmaster
%\item[Freitag, 13. Oktober \Jahr (?), 15 Uhr, Sand]
%   Am Nachmittag des Anfi-Frühstücks wird speziell für neue Masterstudenten ein lockerers Treffen mit Kaffee, Keksen und vielen Informationen zu eurem Studium angeboten.
%   Hier habt ihr die Möglichkeit, Fragen zu stellen, letzte Fragen zu eurem Stundenplan zu klären und schon mal eure Kommilitonen zu treffen, die ihr in den Vorlesungen wiederfinden werdet. \\
%   Die Uhrzeit und der genaue Ort können sich noch ändern, schau daher vorher nochmal auf \url{https://www.fsi.uni-tuebingen.de/erstsemester/veranstaltungen/} nach.
%\fi

%\item[Freitag, 13. Oktober \Jahr, 19:30 Uhr, Neckarmüller]\ \\
%Am Abend des Frühstücks treffen wir uns um 19:30 Uhr bei der Neckarbrücke (\emph{vor} dem Gasthaus „Neckarmüller“) zu einer
%ausgiebigen Kneipentour.

%\textbf{Bushaltestelle:} Neckarbrücke 


% erste Vorlesung
\ifbachelor
\item[Montag, 16. April \Jahr, Morgenstelle]\ \\
Deine erste Vorlesung beginnt um 
\ifwintersemester 8 Uhr – Ihr habt „Mathe I für Informatiker“  \fi
\ifsommersemester 10 Uhr – Ihr habt „Mathe II für Informatiker“  \fi
bei Prof. \Matheprof.
%bei Dr. \Matheprof.
Alles, was du heute (und in Zukunft) benötigst, ist: einen Stift, einen Block und deinen Studierendenausweis.

\textbf{Bushaltestelle:} BG Unfallklinik   
\fi

%fällt im SS weg
\ifkogwiss
\ifbachelor 
 \item[Montag, 16. Oktober \Jahr, 18 Uhr, Psychologisches Institut, Hörsaal]\ \\
 An diesem Abend werdet ihr von Frau Prof. Rolke begrüßt. Zudem stellen sich euch, nach einer kurzen Einführung in den Studienaufbau, die Lehrstühle der Kognitionswissenschaft mit ihren Themen und Forschungsgebieten vor.
 \fi
 \ifmaster
 \item[Montag, 16. Oktober \Jahr, 18:30 Uhr, Psychologisches Institut, Hörsaal]\ \\
 Heute Abend stellen sich euch die verschiedene Lehrstühle der Kognitionswissenschaft vor, um einen Einblick in ihre Forschungsthemen zu ermöglichen.
 \fi
 
 \textbf{Bushaltestelle:} Hölderlinstraße / Uni-Kliniken Tal
\fi 

\ifkogwiss
\ifmaster
\item[Montag, 16. Oktober \Jahr, Ort und Uhrzeit folgen]\ \\
Hier bekommen speziell Master-Studenten eine separate Begrüßung durch Frau Prof. Rolke, die euch einen ersten Einblick ins Studium gibt, indem sie den regulären Studienaufbau des Master Kognitionswissenschaft erläutert. Ort und Zeit werden noch auf \url{https://www.fsi.uni-tuebingen.de/erstsemester/veranstaltungen/} bekannt gegeben.
\fi
\fi 


\item[Dienstag, 17. April \Jahr, 19:00 Uhr, Sand 13 Raum A301]\ \\
Wir möchten euch zu einem kleinen Spieleabend mit guter Gesellschaft und ruhiger Atmosphäre auf dem Sand einladen. Details werden noch auf \url{https://www.fsi.uni-tuebingen.de/erstsemester/veranstaltungen/} bekannt gegeben.

\textbf{Bushaltestelle:} Sand Drosselweg (Rest ausgeschildert) 



%\item[Freitag, 14. Oktober 2016, 13:30 Uhr, Sand]\ \\
 %  Um 13:30 Uhr treffen wir uns auf dem „Sand“ (dem Sitz des Wilhelm-Schickard-Instituts) vor dem Haupteingang
%   wieder. %Dort werdet Ihr gegen 14 Uhr durch Professor \Infoprof, Euren Informatik-I-Dozenten, begrüßt.
   % Nachdem Ihr über den Verlauf des Studiums in den ersten Wochen, Monaten und Semestern informiert wurdet,
  % Hier wird es ebenfalls eine Führung durch das Gebäude geben, bei der die verschiedenen Fachbereiche einen Einblick in ihre Arbeit geben. 
 %  Außerdem werden den Informatikstudenten mögliche Nebenfächer vorgestellt, zusätzlich werden die Studierenden der spezielleren Studiengänge noch einmal gesondert von ihren Ansprechpartnern begrü{\ss}t.
   
  % \textbf{Bushaltestelle:} Sand Drosselweg (Rest ausgeschildert)
  
%\item[Mittwoch, 13. April 2016, 13:37 Uhr, Sand]
   %An diesem Nachmittag wollen wir euch zu einem Workshop Nachmittag auf den Sand einladen. In kleineren Gruppen wollen wir euch verschiedene Programme, Tools und anderes zeigen, was euch im Studium nützlich sein könnte.
   
  %\textbf{Bushaltestelle:} Sand Drosselweg (Rest ausgeschildert)


%\item[Error 408: Request Time-out]\ \\
%Wie im letzten Semester würden wir auch gerne dieses mal wieder Workshops für euch anbieten. Da uns zum jetzigen Zeitpunkt leider noch kein Termin vorliegt, können wir euch nur kurzfristig über unsere Website informieren. Wenn ihr allerdings ein Programm/Thema habt, dass euch besonders am Herzen liegen würde, könnt ihr uns gerne an fsi@fsi.uni-tuebingen.de schreiben. 

%TODO erneuern
\item[Donnerstag, 19. April \Jahr, 19 Uhr, Sand]\ \\
An diesem Donnerstag möchten wir euch zunächst auf den Sand einladen, um in gemütlicher Runde mit anderen Kommilitonen und Fachschaftlern einige Getränke zu konsumieren. Sobald die Zeit ausreichend fortgeschritten ist, möchten wir zusammen mit euch das Clubhausfest (im Clubhaus, Wilhelmstr. 30 direkt gegenüber der Neuen Aula) besuchen. Dieses findet während der Vorlesungszeit jeden Donnerstag statt. Das Besondere daran ist, dass es jede Woche von einer anderen Fachschaft oder studentischen Gruppe organisiert wird und dementsprechende Abwechslung bietet -- ein Besuch lohnt sich also immer. Außerdem ist der Eintritt frei, ihr benötigt lediglich euren Studentenausweis.

\textbf{Bushaltestelle:} Sand Drosselweg (Rest ausgeschildert)


\item[Donnerstag, 17. Mai \Jahr, 21 Uhr]\ \\
Das anfängliche Chaos des Vorlesungsbeginns hat sich gelegt und auch an diesem Donnerstag findet das Clubhausfest statt. Heute lohnt sich der Besuch jedoch ganz besonders, denn die Fachschaften Informatik, Kogni und Psychologie sind Gastgeber!\\ 
 
 
\end{description}
